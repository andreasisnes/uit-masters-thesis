
% Introduction (general motivation for your work, context and goals)
% Context: make sure to link where your work fits in
% Problem: gap in knowledge, too expensive, too slow, a deficiency, superseded technology
% Strategy: the way you will address the problem
\chapter{Introduction}\label{ch:introduction}\glsresetall
The value of cryptocurrencies has rapidly increased on the last years. In 2018 cryptocurrencies had a market capitalization of around \$$300$ billion according to \href{https://coinmarketcap.com/}{CoinMarketCap}, making it comparable to Denmark's \ac{gdp}\cite{P&D_to_the_moon}. Despite the high market capitalization, these cryptocurrencies are mostly unregulated, including the investment platforms called exchanges where investors trade cryptocurrencies and fiat money\footnote{Money made by the government\cite{fiat}}. Due to the anonymity and lack of regulation, this ecosystem has become an appealing field for conducting illegal activities like terrorism, money laundering, customer theft, and fraud~\cite{bitcoin_regulation}.

Exchanges play a central role as they are popular among investors and carry out $99\%$ of all cryptocurrency transactions~\cite{coinsutra}. Unsurprisingly that makes them vulnerable to scammers who seek to pray on the misinformed~\cite{P&D_to_the_moon}. One particular scam that has become popular in cryptocurrency markets over the last few years is the price manipulation scheme \ac{pd}~\cite{P&D_anatomy}. \ac{pd} involves artificially inflating the price of a cheap asset (pump) on an exchange and selling the purchased assets at a higher price. Once the assets are sold off, the price falls (dump) and the affected investors lose their money to those who organized the scam~\cite{P&D_scheme}. Two researchers at the Imperial College London revealed that at least two \ac{pd} schemes are executed daily on a cryptocurrency market, producing roughly \$$7$ million in daily trading volume~\cite{P&D_anatomy}.

As these scammers corrupt exchanges and deceive investors, people are now reluctant to invest in cryptocurrencies due to mistrust and scepticism~\cite{anchor}. In the last two years, a few articles ~\cite{P&D_to_the_moon, P&D_anatomy, P&D_scheme, P&D_pumping} have proposed various methods for detecting \acp{pd}, but none have yet proposed a model that detects \acp{pd} in real-time using deep learning. Detecting \acp{pd} in real-time allows unethical investors to improve upon their trading strategies by having the opportunity of participating in \acp{pd}. But it also allows exchanges to prevent \acp{pd}, making them more trustworthy. 

The incentive of using deep learning is primarily because of the tremendous amount of data cryptocurrency sources continuously produce. \ac{ml} is generally good at solving problems that have large-scale datasets~\cite{aws}. Secondly, detecting \acp{pd} using a rule-based solution is tricky with high-dimensional data, while deep learning has turned out to be very good at it~\cite{lecun2015deep}. Third, deep learning completely outperforms traditional \ac{ml} methods as the scale of data increases~\cite{dl_intrusion, peng2015multi}. 

In this thesis, we present \project, a system that seeks to detect \acp{pd} in real-time using deep learning. \project retrieves, and stores live data seamlessly from multiple cryptocurrency sources. The gathered raw data flows through several enrichment stages to prepare the data for \ac{ml}. In every supervised learning problem, training a model requires prior knowledge of each sample, and because of the infeasibility of manually collecting \acp{pd}, it uses an anomaly detection algorithm to pinpoint suspicious time intervals in historical data that may contain \acp{pd}. To reduce the number of false anomalies, the alleged anomalies goes through a manual filtering process. With both data and labels, we define a labeled dataset to train a model that contains a network of connected serialized layers where each layer incorporates a \ac{ml}.

\section{Problem Definition}\label{sec:problem_definition}
A popular price manipulation scheme carried out on cryptocurrency exchanges is \acp{pd}. We investigate if we can construct a system, named \project, that detects these in real-time using deep learning. Thus, our thesis statement is:

\begin{displayquote}
    \begin{em}
    Real-time classification of \acfp{pd} in cryptocurrencies can be done using deep learning.
    \end{em}
\end{displayquote}


\section{Methodology}\label{sec:methodology}
The final report of the Task Force on the Core of Computer Science\cite{computing_as_a_discipline} define the succeeding three major paradigms as the discipline of computing. The first paradigm, \emph{theory}, is rooted in mathematics and consists of four steps followed in the development of a coherent, valid theory: 
\begin{enumerate}
    \item characterize objects of study (definition);
    \item hypothesize possible relationships among them (theorem);
    \item determine whether the relationships are true (proof); 
    \item interpret results.
\end{enumerate}

The second paradigm, \emph{abstraction} (modeling), is rooted in the experimental scientific method and consists of four stages that are followed in the investigation of a phenomenon: 
\begin{enumerate}
    \item form a hypothesis;
    \item construct a model and make a prediction;
    \item design an experiment and collect data;
    \item analyze results.
\end{enumerate}

The third paradigm, \emph{design}, is rooted in engineering and consists of four steps followed in the construction of a system (or device) to solve a given problem:
\begin{enumerate}
    \item state requirements; 
    \item state specifications;
    \item design and implement the system;
    \item test the system.
\end{enumerate}

This thesis adheres to the paradigms abstraction and design. We investigate our thesis statement's viability through constructing, experimenting, and analyzing. By means of constructing, we design and implement a system \project, that pursue to solve the stated problem. State requirements and specifications advance and change throughout this thesis by experimenting with various designs. Testing the system involves analyzing \project's abilities.

\section{Context}\label{sec:context}
This thesis is written in the context of \href{http://www.corporesano.no/}{Corpore Sano}\footnote{http://www.corporesano.no/}, a center that conducts joint research in the fields sports, medicine, and computer science. Our interdisciplinary research targets elite sports performance development and injury prevention; preventive health care; large-scale population screen; and epidemiological health studies. In the field of computer science, we have a focus on \ac{rd} systems for monitorization, back-end storage, \ac{ml}, and analytics.

Two of our projects involves injury prevention and performance development for the elite soccer players in \ac{til}, our tightly partnered club. With the systems Bagadus~\cite{halvorsen2013bagadus} and Muithu~\cite{johansen2012muithu} currently deployed and used at \acp{til} practices and games. Muithu is a portable video annotation system that integrates real-time coach notations with related video sequences. While Bagadus is a real-time prototype of a sports analytics application, it integrates a sensor system, a soccer analytics annotations system, and a video processing system using a video camera array. A prototype is currently deployed at Alfheim Stadium in Norway, \acp{til} home ground.

In the wake of the interest in cryptocurrencies and blockchain technology, Corpore Sano did a longitudinal study of this ecosystem's most prominent cryptocurrency, Bitcoin~\cite{tedeschi2017trading}. The study investigated how the scalability affects the performance, and how the costs and fees are dependent. The study also proposed two machine learning models that can predict the bandwidth of scheduled transactions according to the fee payers are willing to offer, and the expected revenue for miners according to the time spent mining.

Another blockchain related contribution from Corpore Sano is FireChain~\cite{mikalsen2018firechain}. It combines a byzantine fault-tolerant gossip service and full membership, with a proposal for blockchain systems that does not consume excessive energy. This protocol is building upon FireFlies~\cite{johansen2015fireflies}, an overlay network protocol. The results show that FireChain is feasible, scalable, and use less power than other blockchain related consensus protocols.

\section{Outline}\label{sec:outline}
\begin{itemize}
    \item[] \textbf{\autoref{ch:background}} first describes three well-known software architectures we use throughout this thesis. Then it describes cryptocurrencies and their trading platforms, exchanges, and details about \acp{pd}. Then, it briefly describes deep learning and the structure of an artificial neural network. Finally, it describes some related work in the detection of \acp{pd}. 
    
    \item[] \textbf{\autoref{ch:design}} presents the overview of \project, and details each component, and \project's different phases.
    
    \item[] \textbf{\autoref{ch:implementation}} covers the implementation of \project.
    
    \item[] \textbf{\autoref{ch:evaluation}} evaluates \project's prediction abilities. 
    
    \item[] \textbf{\autoref{ch:conclusion}} summarizes this thesis, presents \project's results and contributions, and outlines future work.
 \end{itemize}