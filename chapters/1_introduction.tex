\chapter{Introduction}\label{ch:introduction}

\section{Problem Definition}\label{sec:problem_definition}

\section{Context}\label{sec:context}
This thesis is written in the context of Corpore Sano Centre\footnote{http://www.corporesano.no/}, a center for sport and health technology.


\newpage\section{Methodology}\label{sec:methodology}
The final report of the Task Force on the Core of Computer Science\cite{computing_as_a_discipline} define the succeeding three major paradigms as the discipline of computing.

The first paradigm, \emph{theory}, is rooted in mathematics and consists of four steps followed in the development of a coherent, valid theory: 
\begin{enumerate}
    \item characterize objects of study (definition);
    \item hypothesize possible relationships among them (theorem);
    \item determine whether the relationships are true (proof); 
    \item interpret results.
\end{enumerate}

The second paradigm, \emph{abstraction} (modeling), is rooted in the experimental scientific method and consists of four stages that are followed in the investigation of a phenomenon: 
\begin{enumerate}
    \item form a hypothesis;
    \item construct a model and make a prediction;
    \item design an experiment and collect data;
    \item analyze results.
\end{enumerate}

The third paradigm, \emph{design}, is rooted in engineering and consists of four steps followed in the construction of a system (or device) to solve a given problem:
\begin{enumerate}
    \item state requirements; 
    \item state specifications;
    \item design and implement the system;
    \item test the system.
\end{enumerate}

\section{Outline}\label{sec:outline}
\begin{itemize}
    \item[] \textbf{\autoref{ch:background}}
    \item[] \textbf{\autoref{ch:design}}
    \item[] \textbf{\autoref{ch:implementation}}
\end{itemize}