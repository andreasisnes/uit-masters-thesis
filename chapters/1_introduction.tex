
% Introduction (general motivation for your work, context and goals)
% Context: make sure to link where your work fits in
% Problem: gap in knowledge, too expensive, too slow, a deficiency, superseded technology
% Strategy: the way you will address the problem
\chapter{Introduction}\label{ch:introduction}\glsresetall
The expense of cryptocurrencies has rapidly increased, and in 2018, cryptocurrencies had a market capitalization of around \$$300$ billion according to \href{https://coinmarketcap.com/}{CoinMarketCap} making it comparable to Denmark's \ac{gdp}\cite{P&D_to_the_moon}. Despite the high market capitalization, these cryptocurrencies are mostly unregulated including the investment platforms called exchanges where investors trade cryptocurrencies and fiat money\footnote{Money made by the government\cite{fiat}}. Due to the anonymity and lack of regulation, this ecosystem has become an appealing field for conducting illegal activities like terrorism, money laundering, customer theft, and fraud~\cite{bitcoin_regulation}.

Exchanges play a central role as they are popular among investors and carry out $99\%$ of all cryptocurrency transactions (trades)~\cite{coinsutra}, unsurprisingly that makes them vulnerable to scammers who seek to pray on the misinformed~\cite{P&D_to_the_moon}. One particular scam that has become popular in cryptocurrency markets over the last few years is the price manipulation scheme \ac{pd}~\cite{P&D_anatomy}. \ac{pd} involves artificially inflating the price of a cheap asset (pump) on an exchange and selling the purchased assets at a higher price. Once the assets are sold off, the price falls (dump) and the affected investors lose their money to those who organized the scam~\cite{P&D_scheme}. Two researchers at the Imperial College London revealed that at least two \ac{pd} schemes are daily executed on a cryptocurrency market, producing \$$7$ million in daily trading volume~\cite{P&D_anatomy}, demonstrating the impact these scammers have.

As these scammers corrupt exchanges and deceive investors, people are now reluctant to invest in cryptocurrencies due to mistrust and scepticism~\cite{anchor}. In the last two years, a few articles ~\cite{P&D_to_the_moon, P&D_anatomy, P&D_scheme, P&D_pumping} have proposed various methods in detection of \acp{pd}, but none have yet proposed a model that detects \acp{pd} in real-time nor a model that uses deep learning. Detecting \acp{pd} in real-time allows unethical investors to improve upon their trading strategies by having the opportunity of participating in \acp{pd}, but it also entitles exchanges to prevent \acp{pd}, making them yet again, more trustworthy. The incentive of using deep learning is first because of the tremendous amount of data cryptocurrency sources continuously produce, as \ac{ml} in general, is good at solving large-scale problems~\cite{aws}. Second, detecting \acp{pd} using a rule-based solution is tricky with high-dimensional data, while deep learning has turned out to be very good at it~\cite{lecun2015deep}. Third, deep learning performs better than traditional \ac{ml} methods~\cite{dl_intrusion, peng2015multi}. 

In this thesis, we present \project, a system that detects \acp{pd} in real-time using deep learning. It retrieves, and stores live data seamlessly from multiple cryptocurrency sources. The gathered raw data then flows through several highly critical enrichment stages before fed into a model that classifies \acp{pd}. In every supervised learning problem, training a model requires prior knowledge (label) of each sample, and because of the infeasibility of manually collecting \acp{pd}, it uses an anomaly detection algorithm to pinpoint suspicious time intervals in historical data. To reduce false positive, the alleged \acp{pd} goes through a manual filtering process. With a labeled dataset, it trains a model that contains a network of connected serialized layers where each layer incorporate a \ac{ml} algorithm.

% This becomes problem when we want to create a dataset containing regular data and \ac{pd} data because of the huge imbalance between the two classes, and \acp{pd} are rare entities hence it is also challenging to obtain a significant amount of \acp{pd} required to train a \ac{ml} model.

% To demonstrate, during this thesis, in a period of $33$ days, \project\space collected a respectively amount of data on approximately $50$GB containing live data from various cryptocurrency sources, where each observation ha time-gap of $5$ seconds. With the collected data, \project generated a dataset containing $43$ features with a time-gap of $5$ seconds between each observation. Since data extracted from 

\section{Problem Definition}\label{sec:problem_definition}
A popular price manipulation scheme carried out on cryptocurrency exchanges is \acp{pd}. We investigate if we can construct a system, named \project, that detects these in real-time using deep learning. Thus, our thesis statement is:

\begin{displayquote}
    \begin{em}
    Real-time classification of \acfp{pd} in cryptocurrencies can be done using deep learning.
    \end{em}
\end{displayquote}

\section{Context}\label{sec:context}
This thesis is written in the context of Corpore Sano Centre\footnote{http://www.corporesano.no/}, a center for sport and health technology.

\section{Methodology}\label{sec:methodology}
The final report of the Task Force on the Core of Computer Science\cite{computing_as_a_discipline} define the succeeding three major paradigms as the discipline of computing.

The first paradigm, \emph{theory}, is rooted in mathematics and consists of four steps followed in the development of a coherent, valid theory: 
\begin{enumerate}
    \item characterize objects of study (definition);
    \item hypothesize possible relationships among them (theorem);
    \item determine whether the relationships are true (proof); 
    \item interpret results.
\end{enumerate}

The second paradigm, \emph{abstraction} (modeling), is rooted in the experimental scientific method and consists of four stages that are followed in the investigation of a phenomenon: 
\begin{enumerate}
    \item form a hypothesis;
    \item construct a model and make a prediction;
    \item design an experiment and collect data;
    \item analyze results.
\end{enumerate}

The third paradigm, \emph{design}, is rooted in engineering and consists of four steps followed in the construction of a system (or device) to solve a given problem:
\begin{enumerate}
    \item state requirements; 
    \item state specifications;
    \item design and implement the system;
    \item test the system.
\end{enumerate}

\section{Outline}\label{sec:outline}
\begin{itemize}
    \item[] \textbf{\autoref{ch:background}}
    \item[] \textbf{\autoref{ch:design}}
    \item[] \textbf{\autoref{ch:implementation}}
    \item[] \textbf{\autoref{ch:evaluation}}
    \item[] \textbf{\autoref{ch:conclusion}}
\end{itemize}